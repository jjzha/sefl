\begin{table*}[t]
\centering
\scriptsize
\renewcommand{\arraystretch}{1.5}
\begin{tabularx}{\textwidth}{|p{3cm}|X|}
\hline
\rowcolor[HTML]{EFEFEF} \textbf{Section} & \textbf{Details} \\ \hline

\textbf{Overview} & 
Your task is to evaluate pairs of feedback responses (Model A and Model B) given to student assignments. You will select which model provides better feedback according to specific criteria.
\newline
\textbf{Key Principles:}
\begin{itemize}
    \item Focus on efficiency and specificity.
    \item Value concise, meaningful feedback over lengthy explanations.
    \item Prioritize direct, actionable suggestions.
    \item Consider both content and delivery.
\end{itemize}
Remember to take breaks; I suggest spending a maximum of 10 minutes per row.
\\ \hline

\textbf{Sheet Information} & 
In the table, pick the one you got assigned. You will see 7 columns and need to fill in columns C and F:
\begin{itemize}
    \item \textbf{Appendix\_assignment:} What the large language model saw when generating an assignment with a possible answer.
    \item \textbf{Assignment:} What the model generated as an assignment and answered.
    \item \textbf{Model A:} Feedback generated by Model A.
    \item \textbf{Model B:} Feedback generated by Model B.
    \item \textbf{Which is better?} The most important part is to evaluate both feedback responses and determine which one is better, based on the assignment and answer.
    \item \textbf{Comments:} Leave comments if needed.
\end{itemize}
\\ \hline

\textbf{Evaluation Criteria} & 
\textbf{Accuracy:} Does the feedback address specific strengths and weaknesses? Are comments relevant to the student work? Is the critique substantive rather than superficial? 
\newline
\textbf{Actionability:} Are suggestions clear and specific? Can students easily understand what to improve? Are recommendations implementable? 
\newline
\textbf{Conciseness:} Is the feedback brief while remaining meaningful? Does it avoid unnecessary elaboration? Is there minimal redundancy? 
\newline
\textbf{Tone:} Is the feedback constructive while being efficient? Does it balance recognition with criticism? Is the language professional? 
\\ \hline

\textbf{Format} & 
\textbf{Preferred Feedback Style:}
\begin{itemize}
    \item Shows good understanding of the concept.
    \item Uses specific examples from the text to support arguments.
    \item Addresses the main question directly.
\end{itemize}
\textbf{Less Preferred Feedback Style:}
\begin{itemize}
    \item Generalized or vague feedback.
    \item Overly verbose or structured responses.
    \item Focuses on theoretical completeness rather than practical advice.
\end{itemize}
\\ \hline

\textbf{Scoring and Pitfalls} & 
\textbf{Scoring:}
\begin{enumerate}
    \item Read the original assignment carefully.
    \item Review both feedback responses.
    \item Evaluate against the criteria.
    \item Select the model that better aligns with the criteria as ``A'' or ``B.''
\end{enumerate}
\textbf{Pitfalls:}
\begin{itemize}
    \item Avoid preferring longer feedback just because it’s lengthy.
    \item Do not choose feedback that only lists general principles.
    \item Avoid letting formatting alone affect your choice.
\end{itemize}
\\ \hline
\end{tabularx}
\caption{Human Annotation Guidelines for Evaluating Assignment Feedback.}
\label{tab:annotation_guidelines}
\end{table*}
